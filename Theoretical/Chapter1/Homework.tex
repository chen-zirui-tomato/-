\documentclass[a4paper]{article}
\usepackage{amsmath}
\usepackage[affil-it]{authblk}
\usepackage{graphicx}

\usepackage[backend=bibtex,style=numeric]{biblatex}

\usepackage{geometry}
\geometry{margin=1.5cm, vmargin={0pt,1cm}}
\setlength{\topmargin}{-1cm}
\setlength{\paperheight}{29.7cm}
\setlength{\textheight}{25.3cm}

\addbibresource{citation.bib}

\begin{document}
% =================================================
\title{Numerical Analysis homework 1}

\author{ZiruiChen 22435036
  \thanks{Electronic address: \texttt{18107732576@163.com}}}
\affil{(24), Zhejiang University }


\date{Due time: \today}

\maketitle

%\begin{abstract}
 %   The abstract is not necessary for the theoretical homework, 
   % but for the programming project, 
    %you are encouraged to write one.      
%\end{abstract}





% ============================================
\section*{I. consider bisection method with the interval[1.5,3.5]}

\subsection*{I-a.what is the width of the interval at nth step?}

width $= 1/2^{n-1}$

\subsection*{I-b.what is the supremum distance between root and midpoint?}

supremum distance is half of the interval at the n-th step $1/2^{n}$

\section*{II. Prove the relationship between the step and relative error.}
Prove:

width of the interval at n-step is $(b_{0}-a_{0}) \cdot \frac{1}{2^{n}}$
so the relative error is $\frac{(b_{0}-a_{0})\frac{1}{2^{n}}}{true value}$,
and obviously, true value $\geq a_{0}$.
now we solve the following inequality 

\begin{align}
  \frac{(b_{0}-a_{0}) \cdot \frac{1}{2^{n}}}{\text{true value}} & \leq \varepsilon \\
  \frac{1}{2^{n}} & \leq \frac{\varepsilon a_{0}}{b_{0}-a_{0}} \\
  2^{n+1} & \geq \frac{b_{0}-a_{0}}{\varepsilon a_{0}} \\
  n+1 & \geq \log_{2}\left(\frac{b_{0}-a_{0}}{\varepsilon a_{0}}\right)
\end{align}

\section*{III.perform iteration of Newton method.}

Give your answers here.

\section*{IV. Prove the error relationship of the Newton method variant.}
Prove:

iterative equation: $x_{n+1} = x_{n} - \frac{f(x_{n})}{f'(x_{0})}$ ,since
\begin{align}
   (x_{n+1}-a) & =C \cdot (x_{n}-a)^{s} \\
  x_{n+1}-a & = x_{n}-a+ \frac{f(x_{n})}{f'(x_{0})} \\
  & = x_{n} - a + \frac{f(x_{n})-f(a)}{f^{'}(x_{0})(x_{n}-a)} \\
  & = (x_{n}-a)(1+\frac{f^{'}(\xi)}{f^{'}(x_{0})})
\end{align}
then we got $s=1$, $c=1+\frac{f^{'}(\xi_{n})}{f^{'}(x_{0})}$


\section*{V. Prove the convergence of Newton method.}
Prove:

sign $g(x) = tan^{-1} = arctanx , x \in (-\frac{\pi}{2},\frac{\pi}{2})$
$g^{1}(x) = \frac{1}{1+x^2} \leq 1 $, so g(x) has fix-point .

if g(x) is converge , $a = g(a) \Rightarrow a=arctan \cdot a \Rightarrow a=0 $.

for any $x_{0} \in (0,\frac{\pi}{2})$, we have $x_{n} \in (0,\frac{\pi}{2})$ ,
for any n $x_{n+1}-x_{n} = arctan \cdot x_{n} - x_{n} < 0 $ , 

so the sequence  $\{x_{n}\}$ is monitoncally decreasing and bounded,converge.

Similarly ,it can be obtained that for any $x_{0} \in (-\frac{\pi}{2},0)$ ,
the sequence $\{x_{n}\}$ is monotonically increasing and bounded, also converge.

if $x_{0}=0$ ,is converge.

To sum up ,$\{x_{n}\}$ is converge for any $x_{0} \in (-\frac{\pi}{2},\frac{\pi}{2})$ .

\section*{VI. Prove that the sequence of values converge.}

the sequence is $x_{1}=\frac{1}{p}, x_{2}=\frac{1}{p+x_{1}},x_{3}= \frac{1}{p+x_{2}},\cdots,x_{n+1}= \frac{1}{p+x_{n}}$.

sign $g(x)=\frac{1}{x+p} ,consider the fix-point of g(x),$ 

then we have $x=\frac{1}{x+p} \Rightarrow x = \frac{-p \pm (p^{2}+4)^{\frac{1}{2}}}{2}$  

$g^{'}(x)<0$ ,g(x) so is monotonically decreasing.

because p>1 ,$x_{1} = \frac{1}{p} \in (0,1) $,use mathematcal induction if $x_{n} \in (0,1)$ ,then $x_{n+1}=\frac{1}{p+x_{n}}$

now we consider the sequence $x_{2n}$ ,$x_{2n}=\frac{1}{p+\frac{1}{p+x_{2n-2}}}$,sign $f(x)=\frac{1}{p+\frac{1}{p+x}}$

then $x_{2n}=f(x_{2n-2})$ ,we can easily see that $x_{2}<x_{4}$ ,assume $x_{2n}<x_{2n-2}$ ,

then $x_{2n+2}-x_{2n}= \frac{1}{p+\frac{1}{p+x_{2n}}} \geq \frac{1}{p+\frac{1}{2x_{2n}}}-x_{2n} \geq 0$

so $x_{2n}$ is monotonically increasing and similarly $x_{2n-1}$ is monotonically decreasing.

because $x_{n}$ > 0,so the sequence $x_{n}$ is converged to $\frac{-p+(p^{2}+4)^{\frac{1}{2}}}{2}$

\section*{VII. what happens in II if $a_{0}<0<b_{0}$?.}
Prove:

We notice that the error is below $\frac{b_{0}-a_{0}}{2^{n+1}}$ ,
\begin{align}
relative \cdot error &= \frac{\|\alpha-x\|}{\|\alpha\|} = \frac{\frac{b_{0}-a_{0}}{2^{n+1}}}{min\{b_{0},a_{0}\}} < \varepsilon \\
& \Rightarrow n+1 \geq \log_{2}\left(\frac{b_{0}-a_{0}}{\varepsilon \cdot min\{b_{0},a_{0}\}}\right) \\
&\Rightarrow n \geq \log_{2}\left(\frac{b_{0}-a_{0}}{\varepsilon \cdot min\{b_{0},a_{0}\}}\right)-1 \\
\end{align}



\section*{VIII. consider multiplicity of root.}
\subsection*{I-a. how can a multiple root be detected?}
Prove:

if $f(x)=0$ has a k multiple root at $x_{0}$ .

the taylor expansion of $f(x)$ around $x_{0}$ is 
\begin{align}
f(x) &= \frac{1}{k!}(x-x_{0})^{k}f^{(k)}(\xi)
\end{align}

then $\lim_{x \to x_{0}} \frac{f(x)}{(x-x_{0})^{i}}$
$
 = \left\{
\begin{array}{ll}
constant & \text{if } i \geq k \\
0 & \text{if } i < k
\end{array}
\right.
$

we just need to check if $\frac{f(x_{0})}{(x-x_{0})^{i}}=0$ or not.

\subsection*{I-b. Prove the quadratic convergence of modification.}
Prove:

if $r$ is a k-th multiple root of $f(x)$ ,

we perform the k-th taylor expansion of $f(x)$ and $f^{'}(x)$around point $r$ ,

\begin{align}
  f(x) &= f(r)+(x-r)f^{'}(r)+\frac{1}{2}(x-r)^2f^{(2)}(r)+\cdots + \frac{1}{(k+1)!}(x-r)^{k+1}f^{(k+1)}(\xi) \\
  f^{'}(x) &= f^{'}(r)+(x-r)f^{(2)}(r) + \frac{1}{2}(x-r)^2f^{(3)}(r)+\cdots+\frac{1}{k!}(x-r)^{k}f^{(k+1)}(\eta) 
\end{align}

then we have $k\cdot \frac{f(x_{n})}{f^{'}(x_{n})}=\frac{k \cdot f^{k+1}(\xi)(x_{n}-r)}{(k+1)\cdot f^{k+1}(\eta)}$ ,so 
\begin{align}
x_{n+1}-r &=x_{n}-r-k\cdot \frac{f(x_{n})}{f^{'}(x_{n})}=x_{n}-r-\frac{kf^{k+1}(\xi)(x_{n}-r)}{(k+1)f^{k+1}(\eta)} \\
\frac{x_{n+1}-r}{x_{n}-r} &= 1-\frac{kf^{k+1}(\xi)(x_{n}-r)}{(k+1)f^{k+1}(\eta)} 
\end{align}

% ===============================================
\section*{ \center{\normalsize {Acknowledgement}} }
Give your acknowledgements here(if any).


\printbibliography

If you are not familiar with \texttt{bibtex}, 
it is acceptable to put a table here for your references.
\end{document}